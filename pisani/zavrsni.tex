\documentclass[times, utf8, zavrsni]{fer}
\usepackage{booktabs}
\begin{document}

% TODO: Navedite broj rada.
\thesisnumber{6351}

% TODO: Navedite naslov rada.
\title{Kontrola ulaza korištenjem beskontaktnih kartica}

% TODO: Navedite vaše ime i prezime.
\author{Filip Ptiček}

\maketitle

% Ispis stranice s napomenom o umetanju izvornika rada. Uklonite naredbu \izvornik ako želite izbaciti tu stranicu.
\izvornik

% Dodavanje zahvale ili prazne stranice. Ako ne želite dodati zahvalu, naredbu ostavite radi prazne stranice.
\zahvala{}

\tableofcontents			

\chapter{Uvod}
U današnjem svijetu pokušavamo povezati sve više stvari, uređaja i pomagala s tehnologijom. Na taj način se pokušava olakšati korištenje i mogućnost automatizacije pomoću jednog centralnog mjesta. Takva rješenja nam omogućuju korištenje jednog uređaja za upotrebu u plaćanju, identifikaciji te mnogim drugim stvarima. Neki od centralnih mjesta su mobilni telefoni te beskontaktne kartice koje se nalaze u džepu većine današnjeg stanovništva.\par
Kontrola ulaza je jedan od poslova koji se od antičkih civilizacija prepuštalo da obavlja čovjek. Vrata koja su koristila mehanizme ključa i brave nisu dopuštale odstupanje od te norme. Tek pojavom kartica s magnetskom trakom došlo je do promjena. Takve kartice dopuštale su da se svakoj osobi dodijeli jedinstveni identifikator. Pomoću čitaća kartica koje su sadržavale spremeljene identifikatore moglo se dopustiti ulaz samo određenim osobama i ujedno voditi evidencija pristupa. Jedna od mana ovakve tehnologije je što korisnik treba karticu dovesti u direktan kontakt s čitaćem te mogućnost jednostavnog repliciranja informacija spremljenih na njima.\par
Tehnologije kao što su radio-frekvencijska identifikacija (\textbf{RFID}) ,nastala 1983. godine, te beskontaktna tehnologija niže frekvencije (\textbf{NFC}), nastala 2003. godine, dozvoljavaju udaljenu komunikaciju između čitaća i kartice ili oznake. U današnje vrijeme zamijenile su upotrebu kartica s magnetskom trakom zbog veće sigurnosti i u slučaju RFID-a mogućnosti za praćenjem položaja kartice ili oznake korištenjem komunikacije velikih frekvencija.\par
Danas se najčešće za kontrolu ulaza koristi beskontaktna tehnologija niže frekvenicije zbog raširenosti u mobilnim telefonima i u slučaju studentske akademske zajednice Republike Hrvatske u njihovim studentskim iskaznicama, što ne zahtjeva uvođenjem posebnih oznaka kao u slučaju radio-frekvencijske komunikacije.

\chapter{Opis problema}

\chapter{Korišteni razvojni alati i biblioteke}

\chapter{Beskontaktna tehnologija niže frekvencije (\textbf{NFC})}

\chapter{Arhitektura i dizajn sustava}

\chapter{Zaključak}
Zaključak.

\bibliography{literatura}
\bibliographystyle{fer}

\begin{sazetak}
Sažetak na hrvatskom jeziku.

\kljucnerijeci{Ključne riječi, odvojene zarezima.}
\end{sazetak}

% TODO: Navedite naslov na engleskom jeziku.
\engtitle{Title}
\begin{abstract}
Abstract.

\keywords{Keywords.}
\end{abstract}

\end{document}
